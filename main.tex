%%%%%%%%%%%%%%%%%%%%%%%%%%%%%%%%%%%%%%%%%
% kaobook
% LaTeX Template
% Version 1.0 (2/2/19)
%
% This template originates from:
% https://www.LaTeXTemplates.com
%
% Authors:
% Federico Marotta (federicomarotta@mail.com)
% Based on the doctoral thesis of Ken Arroyo Ohori (https://3d.bk.tudelft.nl/ken/en)
% and on the Tufte-LaTeX class.
% Modified for LaTeX Templates by Vel (vel@latextemplates.com)
%
% License:
% GPL Version 3 (see included LICENSE file)
%
%%%%%%%%%%%%%%%%%%%%%%%%%%%%%%%%%%%%%%%%%

%----------------------------------------------------------------------------------------
%       PACKAGES AND OTHER DOCUMENT CONFIGURATIONS
%----------------------------------------------------------------------------------------

\documentclass[
        fontsize=10pt, % Base font size
        twoside=true, % Use different layouts for even and odd pages (in particular, if twoside=true, the margin column will be always on the outside)
        %open=any, % If twoside=true, uncomment this to force new chapters to start on any page, not only on right pages
        %chapterprefix=true, % Uncomment to use the word "Chapter" before chapter numbers everywhere they appear
        %chapterentrydots=true, % Uncomment to output dots from the chapter name to the page number in the table of contents
        numbers=noenddot, % Comment to output dots after chapter numbers; the most common values for this option are: enddot, noenddot and auto (see the KOMAScript documentation for an in-depth explanation)
        %draft=true, % If uncommented, images will be replaced by empty boxes
        %overfullrule=true, % If uncommented, overly long lines will be marked by a black box
]{kaobook}

% Load common packages and commands
\usepackage{styles/environments}
\usepackage{styles/mdftheorems}
%\usepackage{styles/plaintheorems}

% Load packages for testing
\usepackage{blindtext}
\usepackage{zhlipsum}
%\usepackage{showframe}
%\usepackage{showlabels}
\usepackage{calc}


\graphicspath{{images/}{./}} % Paths in which to look for images

\addbibresource{main.bib} % Bibliography file

\makeindex[columns=3, title=按字母排序的索引, intoc] % Create an index

\makeglossaries % Create a glossary

\makenomenclature % Create nomenclature

%----------------------------------------------------------------------------------------
\renewcommand\proofname{证明}
\begin{document}

%----------------------------------------------------------------------------------------
%       BOOK INFORMATION
%----------------------------------------------------------------------------------------

\titlehead{\texttt{kaobook}类}
\subject{使用此文档作为模板}

\title[示例及说明文档 {\normalfont\texttt{kaobook}} 类]{示例及说明文档 \\ of the {\normalfont\texttt{kaobook}} 类}
\subtitle{根据自己需要定制本页}

\author[Federico Marotta]{Federico Marotta \thanks{A \LaTeX\ lover}}

\date{\today}

\publishers{奥色姆曼出版社}

%----------------------------------------------------------------------------------------

\frontmatter % Denotes the start of the pre-document content, uses roman numerals

%----------------------------------------------------------------------------------------
%       OPENING PAGE
%----------------------------------------------------------------------------------------

%\makeatletter
%\extratitle{
%       % In the title page, the title is vspaced by 9.5\baselineskip
%       \vspace*{9\baselineskip}
%       \vspace*{\parskip}
%       \begin{center}
%               % In the title page, \huge is set after the komafont for title
%               \usekomafont{title}\huge\@title
%       \end{center}
%}
%\makeatother

%----------------------------------------------------------------------------------------
%       COPYRIGHT PAGE
%----------------------------------------------------------------------------------------

\makeatletter
\uppertitleback{\@titlehead} % Header

\lowertitleback{
        \textbf{Disclaimer}\\
        你可以编辑这个页面来满足你的需要。例如,这里有一个无版权的声明、一个版权页标记和其他一些信息。这一页是基于肯·阿罗约·奥赫里的论文的相应页面,改动很小。
        
        \medskip
        
        \textbf{No copyright}\\
        \cczero\ This book is released into the public domain using the CC0 code. To the extent possible under law, I waive all copyright and related or neighbouring rights to this work.
        
        To view a copy of the CC0 code, visit: \\\url{http://creativecommons.org/publicdomain/zero/1.0/}
        
        \medskip
        
        \textbf{Colophon} \\
        This document was typeset with the help of \href{https://sourceforge.net/projects/koma-script/}{\KOMAScript} and \href{ttps://www.latex-project.org/}{\LaTeX} using the \href{https://github.com/fmarotta/kaobook/}{kaobook} class.
        
        The source code of this book is available at:\\\url{https://github.com/fmarotta/kaobook}
        
        (You are welcome to contribute!)
        
        \medskip
        
        \textbf{Publisher} \\
        First printed in Jan 2019 by \@publishers
}
\makeatother

%----------------------------------------------------------------------------------------
%       DEDICATION
%----------------------------------------------------------------------------------------

\dedication{
        世界的和谐体现在形式和数量上,自然哲学的心和灵魂以及一切诗歌都体现在数学美的概念上。\\
        \flushright -- D'Arcy Wentworth Thompson
}

%----------------------------------------------------------------------------------------
%       TITLE PAGE
%----------------------------------------------------------------------------------------

% Note that \maketitle will actually print many pages.

% If twoside=false, \uppertitleback and \lowertitleback are not printed. To overcome this issue, we set twoside=semi just before printing the title pages, and set it back to false just after the title pages.
\KOMAoptions{twoside=semi}
\maketitle[3] % The [3] assigns "page 3" to the title, so that the cover page would get "page 1" (see KOMAScript documentation about maketitle)
\KOMAoptions{twoside=false}

%----------------------------------------------------------------------------------------
%       PREFACE
%----------------------------------------------------------------------------------------

% \input{chapters/preface.tex}

%----------------------------------------------------------------------------------------
%       TABLE OF CONTENTS & LIST OF FIGURES/TABLES
%----------------------------------------------------------------------------------------

\begingroup

% Define the style for the TOC, LOF, and LOT
%\setstretch{1}
%\hypersetup{linkcolor=DarkBlue}
\setlength{\textheight}{23cm}

% Turn on compatibility mode for the etoc package
\etocclasstocstyle % "toc display" as if etoc was not loaded
\etocstandardlines % "toc lines as if etoc was not loaded

\tableofcontents % Output the table of contents

\listoffigures % Output the list of figures
% Comment both of the following lines to have the LOF and the LOT on different pages
\let\cleardoublepage\bigskip
\let\clearpage\bigskip

\listoftables % Output the list of tables
\let\cleardoublepage\bigskip
\let\clearpage\bigskip

\listoftheorems

\endgroup

%----------------------------------------------------------------------------------------
%       MAIN BODY
%----------------------------------------------------------------------------------------

\mainmatter % Denotes the start of the main document content, resets page numbering and uses arabic numbers

\setchapterstyle{kao}
\setchapterpreamble[u]{\margintoc}
\chapter{引言}
\labch{intro}

% \section{主要思想}

% 许多现代印刷教科书采用了突出的页边空白处的布局,在这里可以显示小的数字、表格、注释和几乎所有的东西。可以说,这种布局通过将主要文本与辅助材料分离来帮助组织讨论,而辅助材料同时又非常接近文本中引用它的地方。

% 这份文件的目的并不是要道歉,因为有许多更适合这项任务的作者;所有这些单词的目的只是填充空间,以便读者可以看到用kaobook类编写的书是什么样子的。同时,我还将尝试说明类的特性。

% kaobook背后的主要思想来自于这个\href{https://3d.bk.tudelft.nl/ken/en/2016/04/17/a-1.5-column-layout-in latex。html}{blog post},实际上这个类的名称是专门为这篇文章的作者Ken Arroyo Ohori命名的,他允许我根据他的论文创建一个类。因此,如果你想知道更多喜欢1.5栏布局的理由,一定要阅读他的博客文章。

% 您可能已经注意到,灵感的另一个来源是\href{https://github.com/tuft-latex/tuft-latex}{tuft-latex类}。设计相似的原因是很难改进已经很好的东西。但是,我认为这个类比tuft - latex更灵活。例如,我尝试只使用标准包,并尽可能少地从头实现;\sidenote{testteststststtsts}因此,只要您阅读了提供该特性的包的文档,定制任何东西都应该非常容易。

% 在本书中,我将阐述该类的主要特性,并提供有关如何使用和更改内容的信息。让我们开始吧。

% \section{本类的功能}
% \labsec{does}

% \Class{kaobook}类更关注文档结构,而不是样式。实际上,众所周知的\LaTeX\xspace 原则是结构和样式应该尽可能地分离(参见\vrefsec{does})。这意味着这个类将只提供命令、环境和一般情况下的机会来执行用户可能使用或不使用的操作。实际上,类中嵌入了一些样式问题,但是用户可以轻松地定制它们。

% 主要特点如下:

% \begin{description}
% 	\item[Page Layout] 减少文本宽度是为了提高可读性,并为页边距留出空间,以便显示任何类型的元素。
% 	\item[Chapter Headings] 相对于tuft-latex,我们提供了多种章节标题可供选择;例子将在后面的章节中看到。
% 	\item[Page Headers] 它们跨越整个页面,包括页边距,并在双侧模式下交替显示章节和节名。\sidenote[][-2mm]{这是Tufte设计的另一个不同之处。}
% 	\item[Matters] The commands \Command{frontmatter}, 
% 	\Command{mainmatter} and \Command{backmatter} have been redefined in 
% 	order to have automatically wide margins in the main matter, and 
% 	narrow margins in the front and back matters. However, the page 
% 	style can be changed at any moment, even in the middle of the 
% 	document.
% 	\item[Margin text] We provide commands \Command{sidenote} and 
% 	\Command{marginnote} to put text in the 
% 	margins.\sidenote[-2mm][]{Sidenotes (like this!) are numbered while 
% 	marginnotes are not}
% 	\item[Margin figs/tabs] A couple of useful environments is 
% 	\Environment{marginfigure} and \Environment{margintable}, which, not 
% 	surprisingly, allow you to put figures and tables in the margins 
% 	(\cfr \reffig{marginmonalisa}).
% 	\item[Margin toc] Finally, since we have wide margins, why don't add 
% 	a little table of contents in them? See \Command{margintoc} for 
% 	that.
% 	\item[Hyperref] \Package{hyperref} is loaded and by default we try 
% 	to add bookmarks in a sensible way; in particular, the bookmarks 
% 	levels are automatically reset at \Command{appendix} and 
% 	\Command{backmatter}. Moreover, we also provide a small package to 
% 	ease the hyperreferencing of other parts of the text.
% 	\item[Bibliography] We want the reader to be able to know what has 
% 	been cited without having to go to the end of the document every 
% 	time, so citations go in the margins as well as at the end, as in 
% 	Tufte-Latex. Unlike that class, however, you are free to customise 
% 	the citations as you wish.
% \end{description}

% \begin{marginfigure}[-5.5cm]
% 	\includegraphics{monalisa}
% 	\caption[The Mona Lisa]{The Mona Lisa.\\ 
% 	\url{https://commons.wikimedia.org/wiki/File:Mona_Lisa,_by_Leonardo_da_Vinci,_from_C2RMF_retouched.jpg}}
% 	\labfig{marginmonalisa}
% \end{marginfigure}

% The order of the title pages, table of contents and preface can be 
% easily changed, as in aly \LaTeX\ document. In addition, the class is 
% based on \KOMAScript's \Class{scrbook}, therefore it inherits all the 
% goodies of that.

% \section{本类未实现的功能}
% \labsec{doesnot}

% As anticipated, further customisation of the book is left to the user. 
% Indeed, every book may have sidenotes, margin figures and so on, but 
% each book will have its own fonts, toc style, special environments and 
% so on. For this reason, in addition to the class, we provide only 
% sensible defaults, but if these features are not nedded, they can be 
% left out. These special packages are located in the \Path{style} 
% directory, which is organised as follows:

% \begin{description}
% 	\item[style.sty] 这个包包含页面布局、页眉和页脚、章节标题和整个文档中使用的字体的规范。
% 	\item[packages.sty] 加载额外的包,用特殊的内容来装饰写作(例如,这里加载\Package{listing}包,因为不是每本书都需要它)。还定义了一些有用的命令,用于以相同的方式打印相同的单词,例如斜体的拉丁单词或逐字的\Package{packages}。
% 	\item[references.sty] 一些有用的命令来管理标签和引用,再次确保以一致的方式引用相同的元素。
% 	\item[environments.sty] 提供特殊的环境,比如框。简单和复杂的环境都是可用的;所谓复杂,我们的意思是它们被赋予一个计数器,浮动的,可以放在一个特殊的目录中。\sidenote[-2mm][]{参考 
% 	\vrefch{mathematics}来获取更多示例。}
% 	\item[theorems.sty] The style of mathematical environments. 
% 	Acutally, there are two such packages: one is for plain theorems, 
% 	\ie the theorems are printed in plain text; the other uses 
% 	\Package{mdframed} to draw a box around theorems. You can plug the 
% 	most appropriate style into its document.
% \end{description}

% \marginnote[2mm]{The audacious users might feel tempted to edit some of 
% these packages. I'd be immensely happy if they sent me examples of what 
% they have been able to do!}

% In the rest of the book, I shall assume that the reader is not a novice 
% in the use of \LaTeX, and refer to the documentation of the packages 
% used in this class for things that are already explained there. 
% Moreover, I assume that the reader is willing to make minor edits to the 
% provided packages for styles, environments and commands, if he or she 
% does not like the default settings.


% \pagelayout{wide} % No margins
% \addpart{类选项、命令和环境}
% \pagelayout{margin} % Restore margins

% \input{chapters/options.tex}
% \input{chapters/textnotes.tex}
% \input{chapters/figsntabs.tex}
% \input{chapters/references.tex}

% \pagelayout{wide} % No margins
% \addpart{设计和附加功能}
% \pagelayout{margin} % Restore margins

% \input{chapters/layout.tex}
% \input{chapters/mathematics.tex}

% \appendix % From here onwards, chapters are numbered with letters, as is the appendix convention

% \pagelayout{wide} % No margins
% \addpart{附\ 录}
% \pagelayout{margin} % Restore margins

% \input{chapters/appendix.tex}

%----------------------------------------------------------------------------------------

\backmatter % Denotes the end of the main document content

\setchapterstyle{plain} % Output plain chapters from this point onwards

%----------------------------------------------------------------------------------------
%       BIBLIOGRAPHY
%----------------------------------------------------------------------------------------

% The bibliography needs to be compiled on the command line with 'biber main' from the template directory

\defbibnote{bibnote}{Here are the references in citation order.\par\bigskip} % Prepend this text to the bibliography
\printbibliography[heading=bibintoc, title=参考文献, prenote=bibnote] % Add the bibliography heading to the ToC and set the title of the bibliography

%----------------------------------------------------------------------------------------
%       NOMENCLATURE
%----------------------------------------------------------------------------------------

% The nomenclature needs to be compiled on the command line with 'makeindex main.nlo -s nomencl.ist -o main.nls' from the template directory

\nomenclature{$c$}{Speed of light in a vacuum inertial frame}
\nomenclature{$h$}{Planck constant}

\renewcommand{\nomname}{Notation}
\renewcommand{\nompreamble}{The next list describes several symbols that will be later used within the body of the document.}
\printnomenclature % Output the nomenclature

%----------------------------------------------------------------------------------------
%       GREEK ALPHABET
%       Originally from https://gitlab.com/jim.hefferon/linear-algebra
%----------------------------------------------------------------------------------------

\vspace{3cm}
{\usekomafont{chapter}Greek letters with pronounciation} \\[2ex]
\begin{center}
        \newcommand{\pronounced}[1]{\hspace*{.2em}\small\textit{#1}}
        \begin{tabular}{l l @{\hspace*{3em}} l l}
                \toprule
                Character & Name & Character & Name \\ 
                \midrule
                $\alpha$ & alpha \pronounced{AL-fuh} & $\nu$ & nu \pronounced{NEW} \\
                $\beta$ & beta \pronounced{BAY-tuh} & $\xi$, $\Xi$ & xi \pronounced{KSIGH} \\ 
                $\gamma$, $\Gamma$ & gamma \pronounced{GAM-muh} & o & omicron \pronounced{OM-uh-CRON} \\
                $\delta$, $\Delta$ & delta \pronounced{DEL-tuh} & $\pi$, $\Pi$ & pi \pronounced{PIE} \\
                $\epsilon$ & epsilon \pronounced{EP-suh-lon} & $\rho$ & rho \pronounced{ROW} \\
                $\zeta$ & zeta \pronounced{ZAY-tuh} & $\sigma$, $\Sigma$ & sigma \pronounced{SIG-muh} \\
                $\eta$ & eta \pronounced{AY-tuh} & $\tau$ & tau \pronounced{TOW (as in cow)} \\
                $\theta$, $\Theta$ & theta \pronounced{THAY-tuh} & $\upsilon$, $\Upsilon$ & upsilon \pronounced{OOP-suh-LON} \\
                $\iota$ & iota \pronounced{eye-OH-tuh} & $\phi$, $\Phi$ & phi \pronounced{FEE, or FI (as in hi)} \\
                $\kappa$ & kappa \pronounced{KAP-uh} & $\chi$ & chi \pronounced{KI (as in hi)} \\
                $\lambda$, $\Lambda$ & lambda \pronounced{LAM-duh} & $\psi$, $\Psi$ & psi \pronounced{SIGH, or PSIGH} \\
                $\mu$ & mu \pronounced{MEW} & $\omega$, $\Omega$ & omega \pronounced{oh-MAY-guh} \\
                \bottomrule
        \end{tabular} \\[1.5ex]
        Capitals shown are the ones that differ from Roman capitals.
\end{center}

%----------------------------------------------------------------------------------------
%       GLOSSARY
%----------------------------------------------------------------------------------------

% The glossary needs to be compiled on the command line with 'makeglossaries main' from the template directory

\newglossaryentry{computer}{
        name=computer,
        description={is a programmable machine that receives input, stores and manipulates data, and provides output in a useful format}
}

\newacronym[longplural={Frames per Second}]{fpsLabel}{FPS}{Frame per Second}
\newacronym[longplural={Tables of Contents}]{tocLabel}{TOC}{Table of Contents}

\setglossarystyle{listgroup} % Set the style of the glossary (see https://en.wikibooks.org/wiki/LaTeX/Glossary for a reference)
\printglossary[title=Special Terms, toctitle=List of terms] % Output the glossary, 'title'  is the chapter heading for the glossary, toctitle is the table of contents heading

%----------------------------------------------------------------------------------------
%       INDEX
%----------------------------------------------------------------------------------------

% The index needs to be compiled on the command line with 'makeindex main' from the template directory

\printindex % Output the index

%----------------------------------------------------------------------------------------
%       BACK COVER
%----------------------------------------------------------------------------------------

% If you have a PDF file that you want to use as back cover, uncomment the following lines.

%\clearpage
%\thispagestyle{empty}
%\null%
%\clearpage
%\includepdf{cover-back.pdf}

%----------------------------------------------------------------------------------------

\end{document}

%%% Local Variables:
%%% mode: latex
%%% TeX-master: t
%%% End:
