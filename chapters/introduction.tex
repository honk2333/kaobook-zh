\setchapterstyle{kao}
\setchapterpreamble[u]{\margintoc}
\chapter{引言}
\labch{intro}

% \section{主要思想}

% 许多现代印刷教科书采用了突出的页边空白处的布局,在这里可以显示小的数字、表格、注释和几乎所有的东西。可以说,这种布局通过将主要文本与辅助材料分离来帮助组织讨论,而辅助材料同时又非常接近文本中引用它的地方。

% 这份文件的目的并不是要道歉,因为有许多更适合这项任务的作者;所有这些单词的目的只是填充空间,以便读者可以看到用kaobook类编写的书是什么样子的。同时,我还将尝试说明类的特性。

% kaobook背后的主要思想来自于这个\href{https://3d.bk.tudelft.nl/ken/en/2016/04/17/a-1.5-column-layout-in latex。html}{blog post},实际上这个类的名称是专门为这篇文章的作者Ken Arroyo Ohori命名的,他允许我根据他的论文创建一个类。因此,如果你想知道更多喜欢1.5栏布局的理由,一定要阅读他的博客文章。

% 您可能已经注意到,灵感的另一个来源是\href{https://github.com/tuft-latex/tuft-latex}{tuft-latex类}。设计相似的原因是很难改进已经很好的东西。但是,我认为这个类比tuft - latex更灵活。例如,我尝试只使用标准包,并尽可能少地从头实现;\sidenote{testteststststtsts}因此,只要您阅读了提供该特性的包的文档,定制任何东西都应该非常容易。

% 在本书中,我将阐述该类的主要特性,并提供有关如何使用和更改内容的信息。让我们开始吧。

% \section{本类的功能}
% \labsec{does}

% \Class{kaobook}类更关注文档结构,而不是样式。实际上,众所周知的\LaTeX\xspace 原则是结构和样式应该尽可能地分离(参见\vrefsec{does})。这意味着这个类将只提供命令、环境和一般情况下的机会来执行用户可能使用或不使用的操作。实际上,类中嵌入了一些样式问题,但是用户可以轻松地定制它们。

% 主要特点如下:

% \begin{description}
% 	\item[Page Layout] 减少文本宽度是为了提高可读性,并为页边距留出空间,以便显示任何类型的元素。
% 	\item[Chapter Headings] 相对于tuft-latex,我们提供了多种章节标题可供选择;例子将在后面的章节中看到。
% 	\item[Page Headers] 它们跨越整个页面,包括页边距,并在双侧模式下交替显示章节和节名。\sidenote[][-2mm]{这是Tufte设计的另一个不同之处。}
% 	\item[Matters] The commands \Command{frontmatter}, 
% 	\Command{mainmatter} and \Command{backmatter} have been redefined in 
% 	order to have automatically wide margins in the main matter, and 
% 	narrow margins in the front and back matters. However, the page 
% 	style can be changed at any moment, even in the middle of the 
% 	document.
% 	\item[Margin text] We provide commands \Command{sidenote} and 
% 	\Command{marginnote} to put text in the 
% 	margins.\sidenote[-2mm][]{Sidenotes (like this!) are numbered while 
% 	marginnotes are not}
% 	\item[Margin figs/tabs] A couple of useful environments is 
% 	\Environment{marginfigure} and \Environment{margintable}, which, not 
% 	surprisingly, allow you to put figures and tables in the margins 
% 	(\cfr \reffig{marginmonalisa}).
% 	\item[Margin toc] Finally, since we have wide margins, why don't add 
% 	a little table of contents in them? See \Command{margintoc} for 
% 	that.
% 	\item[Hyperref] \Package{hyperref} is loaded and by default we try 
% 	to add bookmarks in a sensible way; in particular, the bookmarks 
% 	levels are automatically reset at \Command{appendix} and 
% 	\Command{backmatter}. Moreover, we also provide a small package to 
% 	ease the hyperreferencing of other parts of the text.
% 	\item[Bibliography] We want the reader to be able to know what has 
% 	been cited without having to go to the end of the document every 
% 	time, so citations go in the margins as well as at the end, as in 
% 	Tufte-Latex. Unlike that class, however, you are free to customise 
% 	the citations as you wish.
% \end{description}

% \begin{marginfigure}[-5.5cm]
% 	\includegraphics{monalisa}
% 	\caption[The Mona Lisa]{The Mona Lisa.\\ 
% 	\url{https://commons.wikimedia.org/wiki/File:Mona_Lisa,_by_Leonardo_da_Vinci,_from_C2RMF_retouched.jpg}}
% 	\labfig{marginmonalisa}
% \end{marginfigure}

% The order of the title pages, table of contents and preface can be 
% easily changed, as in aly \LaTeX\ document. In addition, the class is 
% based on \KOMAScript's \Class{scrbook}, therefore it inherits all the 
% goodies of that.

% \section{本类未实现的功能}
% \labsec{doesnot}

% As anticipated, further customisation of the book is left to the user. 
% Indeed, every book may have sidenotes, margin figures and so on, but 
% each book will have its own fonts, toc style, special environments and 
% so on. For this reason, in addition to the class, we provide only 
% sensible defaults, but if these features are not nedded, they can be 
% left out. These special packages are located in the \Path{style} 
% directory, which is organised as follows:

% \begin{description}
% 	\item[style.sty] 这个包包含页面布局、页眉和页脚、章节标题和整个文档中使用的字体的规范。
% 	\item[packages.sty] 加载额外的包,用特殊的内容来装饰写作(例如,这里加载\Package{listing}包,因为不是每本书都需要它)。还定义了一些有用的命令,用于以相同的方式打印相同的单词,例如斜体的拉丁单词或逐字的\Package{packages}。
% 	\item[references.sty] 一些有用的命令来管理标签和引用,再次确保以一致的方式引用相同的元素。
% 	\item[environments.sty] 提供特殊的环境,比如框。简单和复杂的环境都是可用的;所谓复杂,我们的意思是它们被赋予一个计数器,浮动的,可以放在一个特殊的目录中。\sidenote[-2mm][]{参考 
% 	\vrefch{mathematics}来获取更多示例。}
% 	\item[theorems.sty] The style of mathematical environments. 
% 	Acutally, there are two such packages: one is for plain theorems, 
% 	\ie the theorems are printed in plain text; the other uses 
% 	\Package{mdframed} to draw a box around theorems. You can plug the 
% 	most appropriate style into its document.
% \end{description}

% \marginnote[2mm]{The audacious users might feel tempted to edit some of 
% these packages. I'd be immensely happy if they sent me examples of what 
% they have been able to do!}

% In the rest of the book, I shall assume that the reader is not a novice 
% in the use of \LaTeX, and refer to the documentation of the packages 
% used in this class for things that are already explained there. 
% Moreover, I assume that the reader is willing to make minor edits to the 
% provided packages for styles, environments and commands, if he or she 
% does not like the default settings.
