\setchapterstyle{kao}
\setchapterpreamble[u]{\margintoc}
\chapter{人工智能治理 - 法规与政策}

\section{政策之路}
\subsection{引言}
人工智能治理政策的制定对于引导和规范AI技术的发展至关重要。随着AI技术的快速进步和广泛应用,它对社会的各个方面产生了深远的影响,包括经济结构的变革、就业市场的重塑、个人隐私的保护以及伦理道德的挑战等。在这一过程中,合理的政策制定可以确保AI技术的发展与社会价值观和伦理标准相一致,防止技术滥用和不当行为,同时促进技术创新与社会责任的平衡。此外,政策制定还有助于应对AI带来的安全风险,保护个人隐私和数据安全,维护国家安全和社会稳定。因此,随着AI技术的不断发展,制定和更新相应的治理政策变得尤为紧迫,以确保技术的健康发展和对社会的积极贡献。

\subsection{国内与国外}
中国政府近年来出台了多项⼈⼯智能政策。2017年7⽉,国务院发布了《新⼀代⼈⼯智能发展规划》,旨在推动人工智能技术与经济社会深度融合,并提出到2025年和2030年的发展目标。2017年12月,工信部发布《促进新一代人工智能产业发展三年行动计划》,以新一代人工智能技术的产业化和集成应用为重点,推动人工智能和实体经济深度融合。2019年4⽉,国家市场监管总局发布了《关于加强⼈⼯智能标准化⼯作的指导意⻅》,旨在推动⼈⼯智能标准化⼯作,促进⼈⼯智能技术的健康发展。2021年9月,国家新一代人工智能治理专业委员会发布《新一代人工智能伦理规范》,旨在将伦理道德融入人工智能全生命周期,提供伦理指引。2022年12月,国务院发布《扩大内需战略规划纲要》,明确要在2023-2035年期间大力推动5G、人工智能、大数据等技术与多个领域深度融合。2023年5月,国家互联网信息办公室审议通过《生成式人工智能服务管理暂行办法》,旨在促进生成式人工智能健康发展和规范应用。另外,国务院已将《人工智能法草案》列入立法计划,旨在强化人工智能技术研发,鼓励人工智能前沿研究和原始创新。

国际上对人工智能治理立法也非常重视。欧盟在2016年即通过了《通用数据保护条例》,旨在限制互联网及大数据企业对个人信息和敏感数据的处理,保护数据主体权利。2021年,欧盟发布了《人工智能法案》,对高风险人工智能系统的使用提出了严格要求。同年,美国通过了《2021年国家人工智能倡议法案》,旨在推动人工智能技术的研究和发展;联合国教科文组织通过了《人工智能伦理建议》,也为人工智能发展提供了指导。2022年6月,加拿大出台了《人工智能和数据法案》,旨在规范人工智能系统中的国际和省际贸易和商业。

\begin{table}[htbp]
\begin{tabular}{|l|l|l|l|}
\hline
\textbf{时间} & \textbf{政府/机构} & \textbf{文件/法案} & \textbf{目的} \\
\hline
2016年 & 欧盟 & 通用数据保护条例 & 限制个人信息和敏感数据的处理,保护数据主体权利 \\
\hline
2017年7月 & 国务院 & 新一代人工智能发展规划 & 推动人工智能技术与经济社会深度融合,制定发展目标 \\
\hline
2017年12月 & 工信部 & 促进新一代人工智能产业发展三年行动计划 & 推动人工智能技术产业化和集成应用,促进实体经济深度融合 \\
\hline
2019年4月 & 国家市场监管总局 & 加强人工智能标准化工作指导意见 & 推动人工智能标准化工作,促进技术健康发展 \\
\hline
2021年9月 & 国家新一代人工智能治理专业委员会 & 新一代人工智能伦理规范 & 将伦理道德融入人工智能全生命周期,提供伦理指引 \\
\hline
2021年11月 & 联合国教科文组织 & 人工智能伦理建议 & 提供人工智能发展指导 \\
\hline
2021年 & 欧盟 & 人工智能法案 & 对高风险人工智能系统提出严格要求 \\
\hline
2022年6月 & 加拿大 & 人工智能和数据法案 & 规范人工智能系统中的国际和省际贸易和商业 \\
\hline
2022年12月 & 国务院 & 扩大内需战略规划纲要 & 推动5G、人工智能、大数据与多个领域深度融合 \\
\hline
2023年 & 国务院 & 人工智能法草案 & 强化人工智能技术研发,鼓励前沿研究和原始创新 \\
\hline
2023年5月 & 国家互联网信息办公室 & 生成式人工智能服务管理暂行办法 & 促进生成式人工智能健康发展和规范应用 \\
\hline
2021年 & 美国 & 2021年国家人工智能倡议法案 & 推动人工智能技术的研究和发展 \\
\hline
\end{tabular}
\end{table}


\subsection{支持与约束}

当前国内外关于人工智能治理的法规与政策既包括对技术发展的积极推动,也包括对潜在风险的审慎管理。

中国政府2017年制定的《新⼀代⼈⼯智能发展规划》中提出了到2030年成为世界主要人工智能创新中心的目标,充分体现了中国对AI发展的高度重视和积极支持。在科技部等部门2022年联合印发的《关于加快场景创新以人工智能高水平应用促进经济高质量发展的指导意见》中,也提出了鼓励通过场景创新推动AI技术的应用,促进经济的高质量发展的意见。

在大力支持人工智能发展的同时,相关政策法规也非常关注对人工智能发展进行必要的约束。例如,《中华人民共和国个人信息保护法》即围绕个人信息的处理,确立了处理规则、跨境提供、个人权力、处理者义务等方面的规则,对AI技术在数据收集和处理方面的应用提供了法律约束。2021年出台的《新一代人工智能伦理规范》则针对AI发展提出了增进人类福祉、保护隐私安全、确保可控可信等基本伦理要求,并对特定活动提出了具体的伦理要求。

\subsection{管理与限制}
同时,国内外立法和行业管理机构也非常重视引导AI技术的发展和管控风险。

在欧盟委员会提出的《人工智能法案》中,在建立统一的AI技术规则时,重点强调了风险管理和伦理标准,并对高风险AI技术规定了与之匹配的监管等级。我国在《新⼀代⼈⼯智能发展规划》中,制定了促进AI发展的法律法规和伦理规范,建立了人工智能法律主体及相关权利、义务和责任框架,针对AI可能带来的社会、伦理和法律问题,制定相应的政策措施。

在风险管控方面,我国在《关于加快场景创新以人工智能高水平应用促进经济高质量发展的指导意见》中,努力建立健全公开透明的AI监管体系,实行设计问责和应用监督并重的双层监管结构。欧盟的《人工智能法案》引入了风险分级监管、市场准入制度、监管沙盒等制度;美国政府提出的《人工智能权利法案蓝图》,则为AI系统设立了五项基本原则,包括数据隐私、算法歧视保护等,旨在保护人权并推动AI技术的健康发展。

\section{法律之网}

\subsection{引言}
AI技术的发展为法律领域带来了新的挑战和机遇。一方面,AI的应用可以提高法律服务的效率和质量,例如通过自动化的法律咨询、案例分析和文档审查等。另一方面,AI技术的复杂性和自主性也引发了法律和伦理问题,如数据隐私、算法歧视、责任归属等,这些问题需要法律专家和立法机构制定相应的法规和政策来解决。同时,AI的发展也促使法律体系不断更新,以适应技术变革带来的新情况。法律需要明确AI的权利和责任界限,确保技术的合理使用,并保护个人和社会的利益。因此,AI与法律之间的关系是动态发展的,需要持续的对话和合作,以实现科技与法律的和谐共存。

\subsection{AI系统的法律责任}

AI系统中法律责任的界定问题是一个复杂且不断发展的领域。随着自主决策系统的普及和能力的提升,AI的行为可能引发法律纠纷,这要求法律体系能够适应新技术带来的挑战。

首先,当AI系统的行为引发法律纠纷时,传统的侵权责任法原则可能需要调整以适应AI的特殊性。根据现有的法律框架,AI本身通常不被认为具有法律责任主体的地位,因为它们缺乏承担责任所需的财产,并且不能具有人类的主观意图或过错。因此,责任通常归属于与AI系统的设计、开发、生产、销售或使用相关的个人或实体。

在判定责任时,法院可能需要考虑多个因素,包括但不限于:

1. 设计缺陷:如果AI系统的设计存在缺陷,导致了损害的发生,设计者可能需要承担责任。
2. 生产缺陷:如果AI系统在生产过程中出现缺陷,生产者可能需要承担责任。
3. 使用不当:如果AI系统的使用者未能正确使用系统,导致了损害,使用者可能需要承担责任。
为了提高法律责任的明确性和可追溯性,以下是一些建议:
4. 透明度和可解释性:AI系统应当设计为能够提供关于其决策过程的透明和可解释的信息,这有助于确定责任归属。
5. 记录和审计:应要求AI系统的开发者和使用者记录和保存系统的操作日志和决策数据,以便在发生纠纷时进行审计和分析。
6. 法律和伦理准则:制定和实施针对AI系统的法律和伦理准则,明确AI系统的行为标准和责任主体的义务。
7. 保险和风险管理:鼓励或要求AI系统的开发者和使用者购买责任保险,以减轻潜在的财务损失,并推动风险管理的最佳实践。
8. 持续的法律研究和改革:随着AI技术的发展,法律体系需要不断更新和改革,以确保责任判定能够适应新的技术现实。

AI系统引发法律纠纷时的责任判定需要综合考虑多个因素,并且需要法律体系、技术开发者和使用者共同努力,以确保责任的明确性和可追溯性。随着AI技术的不断进步,法律制度也必须适应新的挑战,确保公平和正义得到维护。

\subsection{AI技术与隐私}

AI技术对隐私法律提出了重大挑战,特别是在个人数据收集、处理和共享方面。AI系统为了实现智能化功能,通常需要大量的数据输入,这包括个人身份信息、行为习惯、位置数据等敏感信息。这些数据的收集和使用如果没有得到适当的管理和保护,就可能侵犯个人隐私权。

在法律框架下,个人数据的收集和处理受到严格的规范。例如,欧盟的《通用数据保护条例》(GDPR)规定了数据最小化、目的限制、数据准确性、透明性等原则,要求数据控制者在处理个人数据时必须遵守。此外,数据主体拥有知情权、访问权、更正权、删除权等,以保护其个人隐私不受侵犯。

然而,AI技术的发展往往需要跨领域、跨平台的数据整合和分析,这与现有的隐私法律框架存在冲突。为了平衡科技创新和隐私权的冲突,以下是一些可能的措施:

1. 强化数据保护法规:更新和完善隐私法律,确保法规能够适应AI技术的发展,同时保护个人隐私权。例如,可以引入数据保护影响评估(DPIA)等机制,对AI系统的数据使用进行预先评估。
2. 促进技术创新与合规性:鼓励AI开发者采用隐私增强技术(PETs),如匿名化、去标识化等,以减少对个人隐私的依赖。同时,推动建立行业标准和最佳实践,引导AI技术的合规发展。
3. 提高透明度和控制权:确保用户能够了解其数据如何被收集和使用,并提供足够的控制权,让用户能够决定自己的数据是否被用于AI分析。
4. 加强监管和执法:监管机构应加强对AI数据处理活动的监督,确保企业遵守隐私法规。对于违反隐私法律的行为,应采取有效的执法措施,包括罚款和制裁。
5. 公众教育和意识提升:通过教育和宣传活动,提高公众对个人数据隐私的认识,使他们能够更好地保护自己的隐私权益。

显然,平衡科技创新和隐私权的冲突需要多方面的努力,包括法律制定者、技术开发者、企业和用户等各方的参与和合作。只有制定合理的法规、采用创新的技术解决方案、加强监管和提高公众意识,我们才可以在保护个人隐私的同时,促进AI技术的健康发展。

\subsection{透明度与可解释性的法律规定}

法律对AI系统的透明度和可解释性的要求是为了确保AI系统的决策过程能够被解释和审查,满足法律的规制和伦理的要求。

各国政府已经开始立法对AI系统的透明度和可解释性加以规范,以确保用户和监管机构能够理解AI系统的决策过程。例如,欧盟的《通用数据保护条例》(GDPR)要求,在完全自动化的决策过程中,数据主体有权获得决策逻辑的解释。我国的《个人信息保护法》也提出了类似的要求,强调算法自动化决策的透明度和结果的公平、公正。

AI系统的“黑箱”特性使得其决策过程难以解释。为了提高可解释性,研究者和开发者正在探索各种技术手段,如可解释的人工智能(XAI)和算法说明书机制。这些技术旨在提供关于AI决策过程的清晰和有意义的信息,以便用户和监管机构能够进行有效的审查。

伦理原则和法律规制的结合也是提高AI系统透明度和可解释性的关键。例如,我国的《新一代人工智能伦理规范》提出了保护隐私安全、确保可控可信等伦理规范。欧盟的《人工智能法案》也旨在建立一套统一的规范和监管框架,确保AI技术的发展和应用能够遵循公平、透明和可信的原则。

尽管法律和伦理原则提供了指导,但在实践中确保AI系统的透明度和可解释性仍面临挑战。与欧美相比,目前中国的算法治理规则比较分散,缺乏实施的细则和操作指引。为了应对这些挑战,建议采取包容审慎的立场,建立分级分类分场景的监管方式,同时借鉴食品营养成分表等信息披露机制,为符合条件的AI系统建立“算法说明书”机制。

确保AI系统的透明度和可解释性是满足法律要求的关键。这需要法律、技术、伦理和市场力量的共同作用,通过制定合理的法规要求、开发有效的技术解决方案、建立伦理审查机制,并在实践中不断调整和完善相关政策和措施。通过这些努力,可以增进用户对AI系统的信任,防范算法歧视,支持内部治理,并促进人机协作。

 
\subsection{国际合作与法规规定}

国际人工智能公司(例如OpenAI)在遵守不同国家法规时面临的法律挑战主要包括以下几点:

1. 多元法律体系的适应性:跨国公司的业务活动跨越多个司法管辖区,需要遵守各国独立的法律制度和法规框架。不同国家的法律体系差异巨大,这对公司构成了适应性的挑战,要求公司必须熟悉并遵守各国的法律法规,以避免法律风险和罚款。
2. 合规监管的复杂性:跨国公司不仅要遵守所在国的法律法规,还需要考虑国际组织的要求,如世界贸易组织(WTO)、经济合作与发展组织(OECD)等的相关规定。这些监管框架的复杂性使得合规工作变得更加困难。
3. 国际法规协调的努力:当前,国际组织和协定正在努力协调跨境人工智能法规。例如,OECD建立了全球人工智能合作伙伴关系框架,旨在促进国际间的合作与对话,推动形成共同的人工智能治理原则和标准。
4. 国际合作机制的需求:为了应对上述挑战,法规制定中可能需要的国际合作机制包括:
  - 建立国际标准:通过国际标准化组织等机构制定统一的技术标准和规范,降低跨国公司在不同国家运营的合规成本。
  - 促进信息共享:建立国际平台,促进各国监管机构之间的信息共享,提高监管效率和透明度。
  - 共同监管框架:制定跨国界的共同监管框架,为人工智能等新兴技术提供一致的法律指导和监管要求。
  - 国际争端解决机制:设立专门的国际争端解决机构,处理跨国公司在遵守不同国家法规时出现的法律争端。
 
通过这些国际合作机制,可以促进全球法规的一致性和协调性,帮助跨国公司更有效地应对不同国家的法律挑战,同时也有助于保护消费者权益和促进公平竞争。
 

\subsection{法官培训与技术审查}
法官和法律专业人士在处理人工智能相关案件时面临的法律认知问题主要包括对人工智能技术的理解和评估、算法决策的透明度和可解释性、以及技术性证据的审查等方面。

1. 对人工智能技术的理解和评估:法官在裁判过程中需要对涉及人工智能技术的案件进行评估,这要求他们具备一定的技术知识。然而,人工智能技术的复杂性和专业性可能导致法官在理解和评估相关技术时遇到困难,从而影响案件的公正裁决。

2. 算法决策的透明度和可解释性:人工智能系统尤其是那些基于机器学习的系统,往往被视为“黑箱”,其决策过程缺乏透明度。法官在审理案件时,需要能够理解和解释算法决策的逻辑,以确保裁判的公正性和合理性。

3. 技术性证据的审查:随着技术性证据在案件中的重要性日益增加,法官和法律专业人士需要具备审查这些证据的能力。这包括对电子数据、算法生成的报告等技术性证据的真实性、合法性和相关性的评估。

为了更好地处理这些问题,可能需要对法官进行专门的培训,以提高他们对人工智能技术的理解,并培养他们在技术性证据审查方面的能力。这种培训可以包括:

1. 技术知识教育:通过培训,法官可以了解人工智能的基本原理、常见应用以及可能的法律问题。
2. 案例分析:通过分析涉及人工智能的具体案例,法官可以更好地理解技术如何在司法实践中应用。
3. 实践操作:提供模拟操作环境,让法官亲身体验人工智能系统的工作流程,增强对技术的直观理解。

法律系统整合技术审查机制的方法可能包括:

1. 建立专家咨询制度:聘请技术专家作为顾问,为法官提供专业的技术意见。
2. 设立专门的技术审查团队:组建由技术专家组成的团队,专门负责审查技术性证据。
3. 完善技术性证据审查制度:制定明确的技术性证据审查标准和流程,确保审查工作的规范性和有效性。

总之,随着人工智能技术在司法领域的应用日益广泛,法官和法律专业人士需要通过专门的培训和法律系统的技术整合,提升对人工智能相关案件的处理能力,确保司法公正和技术发展的和谐统一。

\subsection{法规的灵活性与科技创新}
法律法规的制定对AI技术的发展和应用至关重要,因此需要不断评估现有法律法规的适用性,尤其是是否足以适应不断变化的AI技术。人工智能技术日新月异,新技术和应用不断涌现,相关法规必须足够灵活,才能及时适应技术的变革,确保法规的相关性和有效性。同时,法规的灵活性更可以鼓励企业和研究机构进行创新尝试,避免因法规僵化而抑制科技创新和产业发展。

另一方面,在面对人工智能技术可能带来的隐私、安全等方面的风险时,灵活的法规能够更好地应对这些风险,保护公众利益和社会稳定。尤其是面对人工智能技术的发展带来的伦理和道德风险时,灵活的法规能够更好地融入伦理道德考量,引导技术发展符合社会价值和伦理标准。

法规的灵活性可以帮助公众建立对人工智能技术的信任,通过合理的监管措施,让公众相信技术的发展是在可控和有序的框架下进行的。是当人工智能技术与其他领域如医疗、交通、教育等融合时,更需要法规能够灵活适应不同领域的特殊需求和挑战。而随着全球人工智能竞争的加剧,灵活的法规在吸引国际投资和顶尖人才进而提升国家在全球市场中的竞争力方面,其作用亦不可小觑。

对现有人工智能法规的灵活性评估需要从多个角度进行考量。从法规制定的速度来看,随着人工智能技术的快速发展,法规的更新和制定必须能够及时跟进技术的步伐。

为了在法规中融入灵活性以促进科技创新,可以采取以下几个策略:

1. 动态监管机制:建议建立动态的人工智能分级分类监管机制,通过区分关键人工智能和一般人工智能,避免“一刀切”的监管,这样可以根据不同技术的特点和发展阶段,实施更为精准和适应性强的监管策略。

2. 预留接口与灵活条款:法规应为未来可能出现的新技术和应用场景预留足够的接口,例如在知识产权问题上留出调整空间,允许法规在未来根据技术发展进行适应性调整。

3. 公众参与和透明度:增加公众参与度,通过征求意见、公众听证会等方式,让法规制定过程更加透明,同时确保公众对新技术的理解和关切能够被立法者听取和考虑。

4. 风险评估与管理:强调基于风险的监管方法,对不同风险级别的人工智能应用采取不同的管理措施,确保监管措施与技术发展的风险相匹配。

5. 国际合作与标准制定:积极参与国际规则和标准的制定,推动国际间的规则互认,以便国内法规能够与国际标准保持一致,促进技术的国际交流和合作。

6. 促进发展创新原则:在法规中明确提出鼓励人工智能研发和应用,支持基础设施建设,推动公共资源的开放共享,创新探索适应人工智能发展的知识产权制度。

通过上述策略,法规可以更好地适应技术的快速发展,同时为科技创新提供支持和空间。这要求立法者、监管者和技术开发者之间建立紧密的沟通和协作机制,确保法规既能够保护公共利益,又能够促进技术的健康发展。另一方面,考虑到人工智能技术的发展速度可能会超过了现有法规制定的速度,特别是在应用场景和商业模式尚不明朗的探索阶段,立法时机可能尚未成熟,过早的立法可能会限制技术创新,所以,对立法时机的选择也需要高度灵活,要不疾不徐。

\subsection{未来的法律框架}
未来人工智能(AI)治理法规的法律框架需要适应技术发展的趋势和特点,同时也要考虑到社会、伦理和安全等多方面的需求,应是一个多元化、动态化、国际化和人性化的体系,可以有效平衡技术创新与社会责任,确保AI技术的健康、安全和可持续发展。

1. 强化伦理原则:随着AI技术的发展,法律框架将更加重视伦理原则的融入,确保AI的发展符合人类的伦理道德标准。例如,应强调AI应用的公平性、透明性和可解释性,以及对个人隐私的保护。
2. 动态监管机制:法律框架将采用更加灵活的监管机制,以适应AI技术的快速发展。这可能包括定期审查和更新法规,以及建立快速响应机制来处理新出现的技术和社会问题。
3. 跨部门协作:AI技术的应用涉及多个领域,因此法律框架鼓励和规范跨部门协作,确保不同领域的法规能够相互衔接,形成统一协调的治理体系。
4. 公众参与:未来的法律框架应更加注重公众参与,通过公开征求意见、举行听证会等方式,让公众对AI治理法规的制定有更多的发言权和参与度。
5. 国际合作:鉴于AI技术的全球性特征,法律框架应强调国际合作的重要性,参与国际标准的制定,推动国际间的法规互认和协调。
6. 促进创新与防范风险并重:法律框架应在促进技术创新和产业发展的同时,加强对潜在风险的预防和控制,确保AI技术的安全可控。
7. 明确责任归属:随着AI系统自主性的提高,法律框架应明确AI系统及其开发者、使用者的责任归属,确保在发生问题时能够追溯并追究责任。
8. 技术中立性原则:法律框架应尽可能保持技术中立,避免对特定技术或应用领域产生不公平的偏见或限制,同时为新技术的发展留出空间。
9. 强化数据治理:数据是AI技术的基础,法律框架应加强对数据收集、处理和使用的规范,确保数据安全和合规使用。
10. 人工智能教育和培训:法律框架应鼓励和规范AI相关的教育和培训,提高公众和专业人员的AI素养,促进社会的适应和接受。

综上,未来的人工智能治理法律法规应当具备足够的创新性和适应性,以保障其能够跟上AI技术的快速发展,并有效应对由此带来的挑战。法律法规应适时修订、动态更新;应保持技术中立;应进行分级分类监管,根据人工智能系统的风险等级和应用领域进行差异化管理;应明确伦理原则,在公平性、透明性、可解释性以及隐私保护等方面提升公众对AI技术的信任和接受度。从根本上,未来的法律法规应能促进产业发展与创新,在制定过程中应增加公众参与度,让法规制定过程更加透明,确保法律法规的公平、合理、适用,能够成为社会和谐进步的助推剂和稳定剂。
