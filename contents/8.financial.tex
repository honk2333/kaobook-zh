\setchapterstyle{kao}
\setchapterpreamble[u]{\margintoc}
\chapter{金融科技的升腾}
近年来,金融领域正在经历着数字化转型的浪潮,这一变革源于人工智能(AI)与各行业的紧密融合。传统金融机构从最初的基本能力探索逐渐发展为深度垂直整合,形成全面应用和持续创新的阶段。这个转型过程得益于金融机构庞大的客户群体、丰富可靠的数据资源以及强大的技术基础,使其成为高质量人工智能应用的理想场景。与此同时,人工智能技术的兴起,尤其是由ChatGPT等技术带来的狂潮,已经开启了新的应用发展时代。在政策引导、技术驱动和行业实践的共同作用下,人工智能技术在金融领域的广泛应用不断推动着金融供给侧结构性改革、增强金融服务实体经济的能力,同时有效地防范和化解金融风险。

在政策和业务需求双管齐下的背景下,人工智能在金融领域的应用呈现出前所未有的活力和潜力,人工智能在金融领域的应用成为一种趋势,改变着金融行业的生态,引领着金融业向数字化、智能化方向迈进。

在政策层面上,以“规范治理+科技与金融深度融合”为导向的扶持政策得到增强,主要由人民银行、银保监会、科技部等主体制定人工智能战略规划和相关政策。这些政策覆盖金融科技标准、数据安全与隐私保护、数据治理与应用、科技与金融场景深度融合等方向,以进一步促进人工智能与金融的融合,并深化人工智能在金融领域的应用。在业务层面上,人工智能与传统金融业务深度融合,为实体金融机构提供了强大的支持。各种人工智能技术,如机器学习、计算机视觉、知识图谱、自然语音处理、智能语音与对话式 AI,正在逐步在金融领域得以应用,解决了传统金融机构面临的运营成本高、供给不足、门槛高、信息不对称、风险评估难等问题。同样,人工智能在支付结算、存贷款与资本筹集、投资管理、市场设施等金融业务上的应用,帮助金融企业提高数字化能力,推动金融转型升级,开拓新的管理和商业模式,提升了服务实体经济的效率和能力。

人工智能技术在金融领域具有广阔的市场前景,近几年金融机构的技术投入增势显著,《金融科技(FinTech)发展规划(2022-2025年)》等相关政策为未来金融科技的重点投入及创新发展指明了方向,并驱动着金融行业数字化进程加速。艾瑞咨询数据显示,2022年中国银行业IT投入3068亿元,预计未来国内银行业IT投入规模仍将以约24.6\%的复合增长率保持高速增长态势,于2025年达到近6000亿元的规模投入。根据IDC数据显示,中国保险业总体IT投入从2018年的226.4亿元增长至2021年354.8亿元,2017-2021年复合增速为12.06\%。根据中国证券业协会数据统计,2017 年 -2020年,证券行业在信息技术领域投入由 159.86 亿元上涨至 262.87 亿元,2021年全行业投入338.2亿元,同比增长28.7\%。

这一系列数字清晰地表明,金融机构对人工智能技术的投入不断攀升,主要集中在夯实金融科技基础设施、前沿科技采购以及各个领域的技术资金投入。这种趋势不仅反映了金融机构对数字化转型的紧迫需求,同时也预示着人工智能在金融领域持续发展的巨大潜力。在这一过程中,金融机构将更加深度融合人工智能技术,推动金融市场进一步向智能化、数字化转型。


我国在AI+金融技术领域不断迭代创新,积极推动金融行业的数字化进程。在商业模式的解构中,我们可以将整个产业链划分为上游(基础层)、中游(技术层)和下游(应用层)三个关键层面。上游部分由综合技术、区块链、云计算、大数据等厂商主导,它们提供着金融行业数字化转型所需的基础设施支持。这些公司通过不断创新,为整个行业的数字化提供可靠的技术基石。中游则由各类技术公司主导,它们专注于提供人工智能算法、机器学习、生物特征识别、知识图谱等核心技术和解决方案。这一层面的公司在推动技术前沿的同时,也为下游应用层提供了关键的技术支持。在下游,这些先进技术得以广泛应用于金融业务领域,应用场景的多样性不断扩展。从智能营销、智能识别、智能投顾、智能风控到智能客服,这些应用不仅涵盖面广,而且深刻改变了金融服务的形态。金融机构通过引入这些先进技术,提升了运营效率、优化了风险管理,并拓展了服务范围,使得金融科技的融合更加贴近实际业务需求。

\section{信贷评估:AI重塑借贷关系}
\subsection{传统信贷评估的挑战}

在金融领域,信贷评估一直是至关重要的环节,它直接影响着金融机构与借款人之间的借贷关系。传统的信贷评估方法主要依赖于人工进行风险评估和信用调查,然而这种方法面临着一系列挑战。首先,信息不对称,即金融机构难以获取到借款人的全面信息,导致评估结果可能不准确。其次,评估标准的不一致性,不同金融机构的信贷评估标准存在差异,导致对借款人信用风险的评估结果不一致。此外,传统的信贷评估方法效率低下,因为需要大量的人力和时间,已经无法满足日益增长的信贷需求。最后,公平性问题也是一个重要的挑战,由于人为因素的介入,传统信贷评估可能存在主观偏见,导致借款人在获得融资时面临不公平的情况。

随着人工智能技术的迅猛发展,AI正逐渐成为重塑信贷评估方式的关键力量。传统的信贷评估方式往往依赖于人工进行风险评估和信用调查,存在着信息获取不足、评估标准不一、效率低下等问题。而AI技术的应用,通过大数据分析、机器学习和深度学习等手段,能够快速准确地评估借款人的信用风险,为金融机构提供更精准、高效的信贷决策支持。AI重塑了传统的借贷关系,不仅提高了风险控制能力,还为借款人提供了更便捷、公平的借贷服务。在数字化时代,AI信贷评估将成为金融行业不可或缺的一部分,为借贷关系注入更多科技创新的活力。
\subsection{人工智能在信贷评估中的应用}

人工智能技术的发展为改善信贷评估提供了新的解决方案。通过分析大数据和应用机器学习算法,人工智能可以更准确地评估借款人的信用风险,提高信贷市场的效率和公平性。

首先,通过利用大数据技术,人工智能可以收集和分析海量的借款人数据,包括个人信息、财务状况和信用历史。这种深度分析使得人工智能能够挖掘出隐藏在数据背后的规律和趋势,从而更准确地评估借款人的信用风险。其次,借助于训练机器学习模型,人工智能能够从历史数据中学习和识别不同借款人的信用特征,并预测借款人未来的信用表现。决策树、神经网络和支持向量机等机器学习算法的应用使得金融机构更好地理解借款人的信用状况,从而做出更准确的信贷决策。

智能风险评估是人工智能在信贷领域的又一重要应用。基于大数据分析和机器学习算法,人工智能可以实现智能化的风险评估。通过对借款人的个人信息、财务状况、信用历史等多方面数据的综合分析,人工智能可以生成准确的信用评分,帮助金融机构快速、准确地评估借款人的信用风险。这种智能化的风险评估方法不仅提高了评估的准确性,还提高了评估的效率,有助于金融机构更好地管理信贷风险。最后,人工智能不仅能够提供智能化的决策支持,帮助金融机构制定更合理的信贷政策和策略,还能为金融机构提供针对不同借款人的个性化信贷方案,促进信贷市场的健康发展,提高市场效率和公平性。这种智能化的决策支持有助于金融机构更好地满足借款人的需求,推动信贷市场的良性发展。

人工智能通过大数据分析和机器学习算法,为传统的信贷评估带来了革命性的变革。智能风险评估和智能决策支持等技术的应用,提高了信贷评估的准确性和效率,促进了信贷市场的发展和健康。然而,随着人工智能技术的不断发展,我们也需要加强对其潜在风险的监管和管理,确保人工智能在信贷领域的应用能够为社会经济的可持续发展做出积极贡献。

\subsection{应用案例:蚂蚁借呗}
蚂蚁集团是一家在金融科技领域拥有领先地位的公司,他们利用人工智能技术在信贷评估中取得了令人瞩目的成就。以其旗下的蚂蚁借呗(Ant Credit Pay)为例,这是一款基于大数据和机器学习算法的信贷产品。蚂蚁借呗通过分析用户在支付宝平台上的消费数据、交易记录、社交网络等海量数据,结合深度学习算法,对用户的信用风险进行智能评估。

首先,蚂蚁借呗通过大数据分析,收集和分析用户在支付宝平台上的各种数据,包括消费行为、还款记录、社交网络等。这些数据不仅包含用户的个人信息和财务状况,还包括了用户的消费习惯、社交关系等多方面信息。通过对这些数据的深度分析,蚂蚁借呗可以挖掘出隐藏在数据背后的规律和趋势,更准确地评估用户的信用风险。

其次,蚂蚁借呗应用了机器学习算法进行信用评估。通过训练机器学习模型,蚂蚁借呗可以从历史数据中学习和识别不同用户的信用特征,并预测用户未来的信用表现。借助于这些机器学习算法,蚂蚁借呗可以实现智能化的信用评分,帮助金融机构快速、准确地评估用户的信用风险。

通过大数据分析和机器学习算法的应用,蚂蚁借呗不仅提高了信贷评估的准确性,还提高了评估的效率。用户可以在支付宝平台上快速完成信贷申请,无需填写繁琐的申请表格和提供大量的证明材料。蚂蚁借呗还提供了个性化的信贷方案,根据用户的信用状况和需求,为用户提供最适合的借款方案。这种智能化的信贷评估方法为用户提供了更便捷、更高效的信贷服务,也为金融机构降低了风险,促进了信贷市场的发展。

蚂蚁借呗是蚂蚁集团在信贷评估领域的一项重要创新。通过大数据分析和机器学习算法的应用,蚂蚁借呗实现了智能化的信贷评估,为用户和金融机构提供了更高效、更便捷的信贷服务,促进了信贷市场的发展和健康。

\section{反欺诈与风控:AI守护金融安全}
\subsection{不可或缺的风险管理工具}
在当今数字化时代,金融行业面临着日益复杂和多样化的挑战,其中风险管理成为金融机构不容忽视的核心议题。金融机构面对着来自市场波动、经济变化、法律法规变化等多方面的风险挑战。有效的风险管理不仅能帮助金融机构及时识别、评估和应对这些风险,还能保护资产、维护声誉,并实现可持续的业务增长。通过制定适当的风险管理策略、建立健全的内部控制机制、采用先进的技术工具等措施,金融机构可以最大限度地降低风险带来的损失,确保业务的稳健发展。在当今竞争激烈、风险多变的市场环境中,高效的风险管理不仅是金融机构的责任,也是其长期成功的关键。

人工智能技术的应用正在成为金融机构应对这些挑战的重要策略之一。随着人工智能在反欺诈和风险管理中的应用越来越广泛,越来越多的金融机构正积极构建自己的“AI风控墙”。人工智能能够协助金融机构更加智能地识别和预防欺诈行为,提高风险管理的效率和准确性,从而保护客户的个人信息以及资金安全。通过智能化的数据分析和决策支持,金融机构能够更好地应对风险挑战,增强市场竞争力,促进行业的健康发展。

随着科技的不断进步,人工智能在金融领域的应用将不断扩展和深化。未来,金融机构将继续探索和应用更加先进的人工智能技术,进一步提升风险管理的水平和效能,为客户提供更安全、更可靠的金融服务,推动金融行业持续发展。

\subsection{智能风控:筑牢反欺诈防线}

在信息网络和数字化技术的不断普及的背景下,层出不穷的诈骗手段对人民的财产安全造成了严重威胁。由于人口基数大,对于诈骗犯罪的发现和管理难度也变得十分巨大,因此,人工智能在反欺诈和风险管理中发挥着越来越重要的作用,金融系统协助打击电信网络诈骗犯罪,践行“金融为民”理念刻不容缓。

人工智能可以通过分析大数据来识别潜在的欺诈模式和趋势,利用机器学习算法,系统可以从海量数据中提取特征,发现隐藏的模式,并预测可能的欺诈行为。其次,人工智能还可以实时监测交易和客户行为,识别异常模式和异常交易,从而及时发现并阻止可疑活动。此外,人工智能还能够自动化决策过程,根据预先设定的规则和模型对可疑交易进行风险评估,并采取相应的措施,如暂停交易或发出警报。通过智能化的数据分析和决策支持,有效提高了金融系统的安全性和稳定性,还为客户提供了更加可靠和高效的金融服务。

2022年开始,中信银行切实履行职责,通过技防和人防相结合,深层次筑牢金融反诈防火墙,切实保护金融消费者权益,有力保障了客户资金安全。其推出的“哨兵”智能风控系统可以利用大数据、机器学习和人工智能等先进技术,对银行交易和客户行为进行实时监测和分析,以识别和预防欺诈行为。这种做法不仅可以提高反欺诈的效率,还能降低银行和客户的风险,保护客户的资金安全和个人信息。

随着科技的发展,智能风控系统将不断优化和升级,为银行业提供更加可靠的安全防护。“哨兵”智能反欺诈系统通过在欺诈交易关键链路上引入公安、社交、电信等外部风险信息,利用机器学习和大数据技术,实现风控策略的动态调整,管控措施的梯度化和精准化,强化反诈预警能力与劝阻机制,及时发现潜在受害客户,并通过总分支三级联动响应机制,快速触达客户,第一时间协助客户摆脱诈骗人员的洗脑控制,及时阻止客户转账的行为,避免了客户的损失。

“哨兵”智能反欺诈系统包含三大核心科技助力全链条反诈。一是毫秒级异常交易精准识别。基于中信银行的强大算力,对每一笔交易进行事中侦测,毫秒级识别风险,实时触发差异化的安全机制和管控措施,主动拦截和劝阻被诈客户。二是AI模型精准防控。自主研发社团图谱模型等机器学习模型,大幅提升涉案账户排查准确率和筛查效率;客户行为序列模型智能识别非客户意愿交易,有效防范纯诱导型诈骗。三是账户风险精准分级。基于客户资金情况、交易行为等,建立差异化的个人账户风险评级,合理设置非柜面业务限额等管控措施,做到风险与体验的平衡,尽可能减少对客户交易的干扰。

正是AI助力金融安全,才避免了人民的财产流失,社会才能实现安居乐业。在当今数字化时代,金融领域面临着日益复杂的挑战,包括诈骗、恶意攻击和其他形式的金融犯罪。这些威胁不仅对个人财产构成风险,也对整个社会的经济稳定性和安全性构成威胁。然而,随着人工智能技术的不断发展和应用,金融安全得到了有效的保障。


\subsection{数据挖掘与模型应用}
在信贷领域,如何解读征信数据成为金融机构无法跳过的问题。由于其复杂性和多样性,传统数据处理方式往往难以进行有效分析。为了解决这一难题,度小满智能征信中台采用了大型语言模型和图算法,用于解读征信报告。通过充分挖掘报告中的风险变量,结合大数据、人工智能、云计算等先进技术,提升了金融风控模型的风险区分度,支持度小满各环节业务智能化开展,进而助力金融数字化升级,实现科技赋能实体经济。

在保险领域,苏黎世保险公司在当前竞争激烈的保险市场中,意识到传统的理赔和风险管理方法已经无法满足日益增长的客户需求和市场挑战。因此,他们积极探索如何借助最新的人工智能技术来提升其业务水平和竞争力。

苏黎世保险公司正致力于利用ChatGPT人工智能技术。这一先进技术能够分析大量的历史理赔数据,并从中找出具体的损失原因,为保险公司提供更准确、更高效的理赔处理流程。借助深度学习模型,他们能够更加精确地识别出潜在的风险因素,进而改善承保流程,提高理赔的效率和准确性。

随着金融行业的数字化转型加速推进,人工智能技术已经成为各大金融机构提升竞争力的重要利器。各大金融机构不仅投入大量人力、物力和财力发展人工智能技术,还将其广泛应用于反欺诈和风险管理等关键领域。这些努力不仅能够提高公司自身的竞争力,也能够为社会的积极健康发展做出积极贡献。

通过不断探索和创新,金融机构正在逐步打造一个智能化、高效率的服务体系,为客户提供更优质、更便捷的服务。同时,他们也为整个行业的发展注入了新的活力和动力。在未来,随着人工智能技术的不断发展和应用,我相信会有更多的创新应用,让风险管理不再是难题,金融安全得到更加全面的保障,为人民的福祉和社会的稳定作出更大贡献。

\section{智能交易:AI驰骋金融战场}
\subsection{机器学习驱动的量化投资}
每每提及智能交易,人们总是会畅想通过人工智能掌握股市走向,这时就要使用量化投资了。首先介绍一下什么是量化,量化投资是一种基于数学、计算机技术分析市场数据的投资方法,起源于20世纪50年代,核心是使用数学模型和算法确定投资机会的策略,旨在消除情绪影响。其发展可分为三个阶段:统计分析、算法交易和智能投资。统计分析依赖统计学和概率论,探索市场趋势和规律;算法交易利用数理逻辑和AI,实现高速、高效、低成本的交易;智能投资则借助大数据和机器学习,提供智能化、自动化、个性化的投资决策。技术基础在于哈里·马科维茨的现代投资组合理论和布莱克-斯科尔斯的期权定价模型;核心概念包括统计套利(利用均值回归分析投资多元化证券组合)和因子投资(识别影响回报的因子以创建投资组合);代表人物包括查尔斯·道、詹姆斯·西蒙斯和杰夫·贝索斯。

在其中,机器学习在量化投资中发挥至关重要的作用,强化数据驱动策略,识别市场模式与投资信号。它应用于因子研究和模型开发,构建解释股票表现的模型,并动态学习因子关系。机器学习在处理大规模数据和预测股权回报方面具有优势,并显著提高时间序列的可预测性。然而,实施金融机器学习时可能面临挑战,如选择和实施正确算法。解决挑战的关键在于避免常见陷阱,雇佣经验丰富的数据科学家或使用自动化工具。总的来说,机器学习有望提高量化投资策略水平,但需谨慎处理潜在挑战。

除此之外,随着硬件设备与网络建设的发展,一瞬间的延迟就会导致错过市场的变化,于是高频交易应运而生,人工智能也持续助力发展,强化学习成为关键。

\subsection{强化学习在高频交易中的应用}
金融产品追求低买高卖,计算机缩短了交易时间,催生了高频交易(HFT),利用高速算法,低隔夜持仓,直连交易所。投资策略包括套利、做市和趋势/事件驱动,利用强化学习处理高维、动态、不确定环境,从高频数据提取特征和信号,制定有效策略,并在交易中不断调整和改进以提高收益和稳定性。高频交易基站常建在交易所周边,追求极致速度。

强化学习模型在高频交易框架中的要素包括环境、状态、动作和奖励。环境模拟股票市场,包括限价订单簿(LOB)和市场订单。状态由基本、高级和深度特征组成。动作包括买入、卖出、持有和清仓,通过离散建模表示。奖励函数有简单和复杂两种,考虑每步收益和交易成本。为输出投资策略,使用深度神经网络表示,通过近端策略优化(PPO)算法进行训练。模型在高频、非平稳市场中学习和优化策略。最新研究表明基于POMDP的交易框架具有较强的抗风险和盈利能力。利用比特币K线数据进行数据清洗和因子提取,联合使用Q学习、进化策略和策略梯度算法进行训练。实验结果显示进化策略在不同市场情况下获得了显著的收益,优于Q学习和策略梯度算法。

\subsection{市场分析}
上文中我们阐述了相关技术与当前投资方法与策略的联系,那么人工智能的引入为整个市场带来了什么呢?

人工智能(AI)交易通过自动化研究和数据驱动决策,提高了投资者生产力。一项调查显示,使用算法交易的交易者生产力提高了10\%。传统投资公司可能有数百名从业人员,但AI交易技术能够复制一些重复性任务,减少管理费用。人工智能算法可以连续工作,每天24小时监控股票市场。

自90年代以来,人工智能被用于预测股市,主要基于技术分析和基本分析。机器学习模型可以类似于技术分析,通过历史价格数据找到逻辑模式。在基本分析方面,机器学习模型可以通过分析新闻和社交媒体预测股票价格基本指标的影响。

随着人工智能应用的广泛普及,有人提出问题:“人工智能是否能战胜股市?”根据一份德勤报告,美国约70\%的交易通过AI或机器学习完成,尤其是2020年由AI驱动的算法交易占了美国股票交易的一半。普华永道预测,到2030年,全球GDP可能增加14\%,为全球经济贡献了15.7万亿美元,其中一部分来自于交易系统的改进。

算法交易市场目前价值为144.2亿美元,预计未来五年将以8.53\%的复合年增长率达到237.4亿美元。市场监控技术被传统交易员用于跟踪交易和投资组合。算法交易应用有望受益于政府规定、交易需求的增加以及市场监控和交易成本的降低。技术革命改变了与世界互动的方式,大型经纪公司和机构投资者使用算法交易降低交易成本。对云解决方案的需求增加也将支持算法交易市场的增长。

\subsection{ 成功案例——从“量化圣手”到GPT}
在智能交易一步步发展的过程中,有许多把握住机遇的从业者从中获得了巨大的回报。詹姆斯·西蒙斯(James Simons)是一位美国数学家、亿万富翁、对冲基金经理和慈善家。他毕业于麻省理工学院,并在1961年获得加州大学伯克利分校的数学博士学位。西蒙斯与陈省身合作创立了陈-西蒙斯规范理论,对数学和物理学产生重要影响。他创立了文艺复兴科技公司,采用数学和统计分析的定量模型进行对冲基金交易。该公司通过招募科学家,建立保密制度,并采用多样化模型和算法实现创新,取得了惊人的投资回报,使其成为当代最伟大的投资者之一。

文艺复兴科技公司使用计算机模型预测金融市场价格变动,不断收集和分析各种数据,从而发现市场中的模式和低效。公司采用高频交易、杠杆和多元化手段放大收益,降低波动性。西蒙斯因此被誉为"量化圣手",超越了其他传奇投资者。

近期,金融领域对于人工智能的应用不断创新,又掀起了一股新的浪潮。Bloomberg推出了BloombergGPT,Man Group发布了ManGPT,而金融领域的开源大型语言模型FinGPT也应运而生。这标志着金融研究和创新的进步,体现了时代与技术的发展。调查显示,交易员普遍计划在2023年使用AI提高投资组合回报率,而一些领先公司已经成功整合AI到交易分析中。FinGPT作为专为金融领域设计的开源模型,随着时间推移不断改进,增强了处理金融数据的能力。这一趋势体现了金融行业对于AI技术的不断追求和应用创新。下一个获得巨大成功的会是谁呢?

\section{智能投顾:AI担任理财顾问}
\subsection{理性的投资顾问}

随着生活水平的不断提高,家庭可支配收入也逐渐增加,这使得家庭的投资需求增加,但是,投资者总是面临着如何高效配置资产的问题,即如何在风险一定的基础上,获得更高的收益。人们总是将资产配置于储蓄和购买理财产品,安全稳健,但收益也低,当人们意识到,适当承担更多的风险,收益也会更高的时候,越来越多的人们选择通过购买基金和股票来提高收益率,然而,市场上无数的股票和基金如何选择,如何配置投入的资金成为一大难题。

筛选有潜力的基金,看涨的股票,需要专业的知识储备及对市场变化的敏锐度,但并非所有投资者都具备这样的条件,人们总是倾向于跟风购买,看到不断上涨的股票,追涨杀跌,殊不知,看上去理性的交易却饱含着不理性的判断,不愿意承担风险的人选择了风险极高的股票。正是这样,智能投顾应运而生。

智能投顾是一种金融服务模式,利用人工智能技术为投资者提供个性化、智能化的投资建议和投资组合管理服务。该服务通过分析投资者的风险偏好、财务目标以及市场情况等信息,能够生成符合投资者需求的投资组合,并实时根据市场波动调整投资策略。智能投顾不仅能释放人力成本,还能降低投资门槛,为不同类型的投资者提供更高效、更准确的投资建议。

Wealthfront是一家智能投顾公司,总部位于加利福尼亚州帕洛阿尔托,属于硅谷核心地带。公司由Andy Rachleff和Dan Carroll于2008年成立,是美国最大的智能投顾平台之一。Wealthfront的主要业务模式为借助于计算机模型和技术,为经过调查问卷评估的客户提供量身定制的资产投资组合建议,包括股票配置、股票期权操作、债权配置、房地产资产配置等,主要客户为硅谷的科技员工,如Facebook、Twitter、Skype等公司的职员。

智能投顾已经在改变传统理财顾问的销售模式。利用互联网大数据,智能投顾系统对用户行为、市场趋势以及各种投资产品进行详细分析,为客户提供多元化的投资组合推荐。这种服务不仅能够避免潜在的利益冲突,同时也能够显著降低投资者的理财成本,并为他们获取更多的投资收益。Wealthfront作为一家领先的智能投顾平台,主要提供自动化的投资组合理财咨询服务。

用户可以通过Wealthfront平台开设和管理账户,同时评估各种投资组合。投资组合包括两大类:有需要纳税的投资组合(适用个人账户、联合账户、信托账户)和退休金投资组合(适用传统IRAs账户、401(K)Rollovers账户、Roth IRAs账户、SEP IRAs账户),资产类别有十一大类:美股、海外股票、新兴市场股票、股利股票、美国国债、新兴市场债券、美国通胀指数化证券、自然资源、房产、公司债券、市政债券。投资组合的载体为指数基金(ETF,全称为Exchange Traded Fund),依据风险容忍度的不同,向投资者推荐的投资计划可能只包括部分类别的资产。

通过分散的投资组合在降低风险的同时不会降低预期收益率,投资者能够在同样的风险水平上获得更高的收益率 ,或者在同样收益率水平上承受更低的风险。平台选择的资产种类多达11类,一方面有利于提高分散化程度,降低风险;另一方面具有不同资产的特性能为用户提供更多的资产组合选择,满足更多风险偏好类型用户的需求。

\subsection{大模型在保险行业中的应用}

保险行业作为数据密集型行业,是大模型技术的最佳应用领域之一。大模型在保险行业的应用涵盖全业务流程,能够帮助保险公司更深入地理解客户需求、优化产品设计、提高风险评估和定价能力、实现精准营销和服务,并提升理赔便捷性,从而降低运营成本、提升市场竞争力和服务效能,进而提升客户体验。

保险销售作为保险业务流程中的关键环节,目前仍主要依赖人工进行。然而,这种模式存在着代理人服务水平不一、人员流动性大等问题。大模型技术的应用有助于提升代理人的销售服务水平和效率。随着大模型技术的不断发展和营销领域数据的积累,基于垂直领域的大模型可用于打造保险销售机器人,通过科技创新重塑保险销售流程,降低人力成本、提升销售效率。

互联网保险公司Lemonade基于GPT-3技术打造了全线上化销售机器人玛雅(AI.Maya),为客户提供保险咨询及报价等服务。当客户有购买保险的意图时,与玛雅进行约两分钟的在线聊天,即可获得保险产品推荐及报价,客户在线支付完成后,即可以全线上化的方式完成Lemonade的投保流程。

玛雅作为一个全线上化销售的保险产品,拥有一系列关键功能,包括信息收集、风险评估与定价、个性化产品建议及服务、保单助手以及理赔辅助。通过信息收集,玛雅获取客户投保标的相关信息,并利用机器学习算法进行风险评估和定价,为客户提供风险分析报告,以辅助承保和保费确定决策。在产品建议和服务方面,玛雅根据客户需求提供个性化的保险产品建议,并协助完成购买流程。同时,玛雅的保单助手利用自然语言处理技术和机器学习算法,帮助客户理解复杂的保险条款和合同细节。在理赔方面,玛雅通过语音识别和自然语言处理技术提高了理赔效率,帮助客户在事故发生后快速完成理赔流程。

此外,玛雅实现了“千人千面”的个性化定价,根据客户住所、房屋信息、信用数据、历史索赔等多方面信息,为不同客户提供个性化的费率。客户还可以在移动端自由选择保单的保障范围和调整免赔额,以获取不同的保费费率。

Lemonade除了销售机器人玛雅外,还推出了理赔机器人吉姆(AI.Jim)。通过科技创新,Lemonade重塑了保险业务流程,取消了传统的经纪人和代理人渠道,实现了投保、理赔、服务等业务流程的全线上化和智能化,节省了大量人力成本。尽管与传统保险公司相比,Lemonade的保险产品保障范围相似,但保单定价却显著更低。

随着科技的不断发展,智能投顾作为一种新兴的理财服务方式,正以其高效、个性化的特点,逐渐改变着传统理财行业的格局。AI作为理财顾问,不仅能够利用大数据和机器学习算法为投资者提供精准的投资组合建议,还能实现全天候、无间断的服务,帮助投资者更好地管理资产、规划财务。随着人们对个性化服务和科技创新的需求不断增加,智能投顾将成为理财领域的重要趋势,为投资者带来更多便利和收益。在未来,随着AI技术的进一步发展和普及,智能投顾将发挥越来越重要的作用,成为投资者理财规划的首选之一。